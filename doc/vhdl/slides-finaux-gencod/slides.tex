\documentclass{beamer}
\usepackage[T1]{fontenc}
\usepackage[french]{babel}
\usepackage[utf8]{inputenc}
\usepackage{amsmath}
\usepackage{hyperref}
\usepackage{verbatim}
\usepackage{graphicx}
\usepackage{fancyvrb}

\hypersetup{
  backref=true, %permet d'ajouter des liens dans...
  pagebackref=true,%...les bibliographies
  hyperindex=true, %ajoute des liens dans les index
  colorlinks=true, %colorise les liens
  breaklinks=true, %permet le retour a la ligne dans les liens trop longs
  urlcolor=blue,  %couleur des hyperliens (blue pour la version web)
  linkcolor=blue, %couleur des liens internes (blue pour la version web)
  citecolor=blue, %couleur des citation (green pour la version web)
  bookmarks=true,  %cree des signets pour Acrobat
  bookmarksopen=true, %affiche completement les signets Acrobat
  %%%%%%%%%%%%%%%%% HYPER TITLE %%%%%%%%%%%%%%%%%%%%%%%%%%
  pdfsubject={Transparents} %document sous Acrobat.
}

\usetheme{boxes}
% \definecolor{MonJaune}{HTML}{FFCC00}
% \setbeamercolor{block title}{fg=MonJaune}

\include{pygments}

\logo{\includegraphics[height=1cm]{gencod}}
% \logo{\insertframenumber/\inserttotalframenumber}

\newcommand{\FIXME}{\textcolor{red}{FIXME}}
\newcommand{\TODO}[1]{(\textcolor{red}{TODO} #1)}
%\newcommand{\TODO}[1]{}
\newcommand{\LANG}{{\sc Heptagon}}
\newcommand{\lucy}{{\sc Lucid Synchrone}}
\newcommand{\lustre}{{\sc Lustre}}
\newcommand{\scade}{{\sc Scade}}
\newcommand{\scadesix}{{\sc Scade~6}}
\newcommand{\minils}{{\sc MiniLS}}
\newcommand{\heptagon}{{\sc Heptagon}}
\newcommand{\obc}{{\sc Obc}}
\newcommand{\minivhdl}{{\sc MiniVHDL}}
\newcommand{\vhdl}{{\sc Vhdl}}

\title{Revue du projet GenCod}
\author{Adrien Guatto et Marc Pouzet}
\date{Mardi 5 Octobre 2010}

\begin{document}

% \maketitle

\begin{frame}[t,plain]
  \titlepage
\end{frame}

\begin{frame}
  \frametitle{Contexte scientifique}

  \begin{block}{Question soulevée dans le cadre du projet GenCod}
    Comment générer une description de matériel à partir de programmes écrits en
    \scade{} ?
  \end{block}

  \pause

  \begin{block}{Approche retenue}
    Étudier la question et proposer une solution dans le cadre d'un langage
    laboratoire simplifié mais suffisamment proche de \scadesix{} pour un
    transfert ultérieur vers celui-ci.
  \end{block}

  \pause

  \begin{block}{Véhicule de l'expérience}
    \heptagon{} est un sous-ensemble de \lucy{}, langage synchrone
    académique (Pouzet et al.) dont les principes de constructions furent repris
    dans \scadesix{}
  \end{block}
\end{frame}

\begin{frame}
  \frametitle{Réalisation}

  \begin{center}
    \includegraphics[scale=0.65]{comparo.pdf}
  \end{center}

  Les traits marquants d'\heptagon{} sont :

  \begin{itemize}
  \item Les programmes sont structurés en noeuds contenants des équations de
    suites ou des automates.
  \item Le processus de compilation vérifie des propriétés de sûreté et raffine
    le programme jusqu'à générer du code impératif (e.g. langage C).
  \item<alert@2-> Il est possible de générer du code VHDL.
  \end{itemize}
\end{frame}

\begin{frame}
  \frametitle{Génération de code VHDL}

  \begin{center}
    \includegraphics[scale=0.4]{sens_traduction.pdf}
  \end{center}

  Pour produire du code VHDL, deux points de départ sont possibles :

  \begin{enumerate}
  \item Le code C produit par le compilateur original (\scade{} ou
    \heptagon{}). Cette approche est celle de la société GeenSoft.
  \item La représentation intermédiaire à flots de données interne au
    compilateur du langage synchrone.
  \end{enumerate}

  \pause

  Nous avons retenu le second cheminement, et l'avons clairement spécifié afin
  de pouvoir envisager une certification DO-178B ultérieure.
\end{frame}

\begin{frame}
  \frametitle{Exemple 1 : compteur - code original}

  \input{compteur}
\end{frame}


\begin{frame}
  \frametitle{Exemple 1 : compteur - exemple de sortie attendue}

  \[
  \begin{array}{r|llllllllllllllllllllllllllll}
    \hline
    tick & t & t & t & f & \dots \\
    \hline
    top & f & f & t & f & \dots \\
    \hline
    compteur(tick,top) & 1 & 2 & 1 & 1 & \dots \\
    \hline
  \end{array}
  \]
\end{frame}


\begin{frame}
  \frametitle{Exemple 1 : compteur - code MiniLS}

  \footnotesize

  \input{compteur_mls}

  Le code n'est plus formé que d'équations.
\end{frame}

\begin{frame}
  \frametitle{Exemple 1 : compteur - code MiniLS sans reset}

  \footnotesize

  \begin{Verbatim}[commandchars=\\\{\}]
\PY{k+kd}{node}\PY{+w}{ }\PY{n}{compteur}\PY{p}{(}\PY{n}{rst\PYZus{}2}\PY{+w}{ }\PY{p}{:}\PY{+w}{ }\PY{k+kt}{bool}\PY{p}{;}\PY{+w}{ }\PY{n}{tick}\PY{+w}{ }\PY{p}{:}\PY{+w}{ }\PY{k+kt}{bool}\PY{p}{;}\PY{+w}{ }\PY{n}{top}\PY{+w}{ }\PY{p}{:}\PY{+w}{ }\PY{k+kt}{bool}\PY{p}{)}
\PY{+w}{ }\PY{+w}{ }\PY{+w}{ }\PY{+w}{ }\PY{+w}{ }\PY{k}{returns}\PY{+w}{ }\PY{p}{(}\PY{n}{o}\PY{+w}{ }\PY{p}{:}\PY{+w}{ }\PY{k+kt}{int}\PY{p}{)}
\PY{k}{let}
\PY{+w}{ }\PY{+w}{ }\PY{n}{o}\PY{+w}{ }\PY{o}{=}\PY{+w}{ }\PY{p}{(}\PY{k}{if}\PY{+w}{ }\PY{n}{top}\PY{+w}{ }\PY{k}{then}\PY{+w}{ }\PY{l+m+mi}{0}\PY{+w}{ }\PY{k}{else}\PY{+w}{ }\PY{p}{(}\PY{k}{if}\PY{+w}{ }\PY{n}{rst\PYZus{}2}\PY{+w}{ }\PY{k}{then}\PY{+w}{ }\PY{l+m+mi}{0}\PY{+w}{ }\PY{k}{else}\PY{+w}{ }\PY{p}{(}\PY{l+m+mi}{0}\PY{+w}{ }\PY{k}{fby}\PY{+w}{ }\PY{n}{o}\PY{p}{)}\PY{p}{)}\PY{p}{)}
\PY{+w}{ }\PY{+w}{ }\PY{+w}{ }\PY{+w}{ }\PY{o}{+}\PY{+w}{ }\PY{p}{(}\PY{k}{if}\PY{+w}{ }\PY{n}{tick}\PY{+w}{ }\PY{k}{then}\PY{+w}{ }\PY{l+m+mi}{1}\PY{+w}{ }\PY{k}{else}\PY{+w}{ }\PY{l+m+mi}{0}\PY{p}{)}
\PY{k}{tel}
\end{Verbatim}


  La réinitialisation logique est explicitée via un paramètre du nœud.
\end{frame}

\begin{frame}
  \frametitle{Exemple 1 : compteur - code MiniLS final}

  \footnotesize

  \begin{Verbatim}[commandchars=\\\{\}]
\PY{k+kd}{node}\PY{+w}{ }\PY{n}{compteur}\PY{p}{(}\PY{n}{rst\PYZus{}2}\PY{+w}{ }\PY{p}{:}\PY{+w}{ }\PY{k+kt}{bool}\PY{p}{;}\PY{+w}{ }\PY{n}{tick}\PY{+w}{ }\PY{p}{:}\PY{+w}{ }\PY{k+kt}{bool}\PY{p}{;}\PY{+w}{ }\PY{n}{top}\PY{+w}{ }\PY{p}{:}\PY{+w}{ }\PY{k+kt}{bool}\PY{p}{)}\PY{+w}{ }\PY{k}{returns}\PY{+w}{ }\PY{p}{(}\PY{n}{o}\PY{+w}{ }\PY{p}{:}\PY{+w}{ }\PY{k+kt}{int}\PY{p}{)}
\PY{k+kd}{var}\PY{+w}{ }\PY{n}{\PYZus{}v\PYZus{}28}\PY{+w}{ }\PY{p}{:}\PY{+w}{ }\PY{k+kt}{int}\PY{p}{;}\PY{+w}{ }\PY{n}{\PYZus{}v\PYZus{}27}\PY{+w}{ }\PY{p}{:}\PY{+w}{ }\PY{k+kt}{int}\PY{p}{;}\PY{+w}{ }\PY{n}{\PYZus{}v\PYZus{}26}\PY{+w}{ }\PY{p}{:}\PY{+w}{ }\PY{k+kt}{int}\PY{p}{;}
\PY{k}{let}
\PY{+w}{ }\PY{+w}{ }\PY{n}{\PYZus{}v\PYZus{}27}\PY{+w}{ }\PY{o}{=}
\PY{+w}{ }\PY{+w}{ }\PY{+w}{ }\PY{+w}{ }\PY{k}{merge}\PY{+w}{ }\PY{n}{top}
\PY{+w}{ }\PY{+w}{ }\PY{+w}{ }\PY{+w}{ }\PY{+w}{ }\PY{+w}{ }\PY{p}{(}\PY{l}{true}\PY{+w}{ }\PY{o}{->}\PY{+w}{ }\PY{p}{(}\PY{l+m+mi}{0}\PY{+w}{ }\PY{k}{when}\PY{+w}{ }\PY{l}{true}\PY{p}{(}\PY{n}{top}\PY{p}{)}\PY{p}{)}\PY{p}{)}
\PY{+w}{ }\PY{+w}{ }\PY{+w}{ }\PY{+w}{ }\PY{+w}{ }\PY{+w}{ }\PY{p}{(}\PY{l}{false}\PY{+w}{ }\PY{o}{->}
\PY{+w}{ }\PY{+w}{ }\PY{+w}{ }\PY{+w}{ }\PY{+w}{ }\PY{+w}{ }\PY{+w}{ }\PY{+w}{ }\PY{p}{(}\PY{k}{merge}\PY{+w}{ }\PY{n}{rst\PYZus{}2}
\PY{+w}{ }\PY{+w}{ }\PY{+w}{ }\PY{+w}{ }\PY{+w}{ }\PY{+w}{ }\PY{+w}{ }\PY{+w}{ }\PY{+w}{ }\PY{+w}{ }\PY{+w}{ }\PY{p}{(}\PY{l}{true}\PY{+w}{ }\PY{o}{->}\PY{+w}{ }\PY{p}{(}\PY{l+m+mi}{0}\PY{+w}{ }\PY{k}{when}\PY{+w}{ }\PY{l}{true}\PY{p}{(}\PY{n}{rst\PYZus{}2}\PY{p}{)}\PY{p}{)}\PY{p}{)}
\PY{+w}{ }\PY{+w}{ }\PY{+w}{ }\PY{+w}{ }\PY{+w}{ }\PY{+w}{ }\PY{+w}{ }\PY{+w}{ }\PY{+w}{ }\PY{+w}{ }\PY{+w}{ }\PY{p}{(}\PY{l}{false}\PY{+w}{ }\PY{o}{->}\PY{+w}{ }\PY{p}{(}\PY{n}{\PYZus{}v\PYZus{}26}\PY{+w}{ }\PY{k}{when}\PY{+w}{ }\PY{l}{false}\PY{p}{(}\PY{n}{rst\PYZus{}2}\PY{p}{)}\PY{p}{)}\PY{p}{)}
\PY{+w}{ }\PY{+w}{ }\PY{+w}{ }\PY{+w}{ }\PY{+w}{ }\PY{+w}{ }\PY{+w}{ }\PY{+w}{ }\PY{+w}{ }\PY{k}{when}\PY{+w}{ }\PY{l}{false}\PY{p}{(}\PY{n}{top}\PY{p}{)}\PY{p}{)}\PY{p}{)}\PY{p}{;}
\PY{+w}{ }\PY{+w}{ }\PY{n}{\PYZus{}v\PYZus{}28}\PY{+w}{ }\PY{o}{=}\PY{+w}{ }\PY{k}{merge}\PY{+w}{ }\PY{n}{tick}\PY{+w}{ }\PY{p}{(}\PY{l}{true}\PY{+w}{ }\PY{o}{->}\PY{+w}{ }\PY{p}{(}\PY{l+m+mi}{1}\PY{+w}{ }\PY{k}{when}\PY{+w}{ }\PY{l}{true}\PY{p}{(}\PY{n}{tick}\PY{p}{)}\PY{p}{)}\PY{p}{)}
\PY{+w}{ }\PY{+w}{ }\PY{+w}{ }\PY{+w}{ }\PY{+w}{ }\PY{+w}{ }\PY{+w}{ }\PY{+w}{ }\PY{+w}{ }\PY{+w}{ }\PY{+w}{ }\PY{+w}{ }\PY{+w}{ }\PY{+w}{ }\PY{+w}{ }\PY{+w}{ }\PY{+w}{ }\PY{+w}{ }\PY{+w}{ }\PY{+w}{ }\PY{+w}{ }\PY{p}{(}\PY{l}{false}\PY{+w}{ }\PY{o}{->}\PY{+w}{ }\PY{p}{(}\PY{l+m+mi}{0}\PY{+w}{ }\PY{k}{when}\PY{+w}{ }\PY{l}{false}\PY{p}{(}\PY{n}{tick}\PY{p}{)}\PY{p}{)}\PY{p}{)}\PY{p}{;}
\PY{+w}{ }\PY{+w}{ }\PY{n}{o}\PY{+w}{ }\PY{o}{=}\PY{+w}{ }\PY{n}{\PYZus{}v\PYZus{}27}\PY{+w}{ }\PY{o}{+}\PY{+w}{ }\PY{n}{\PYZus{}v\PYZus{}28}\PY{p}{;}
\PY{+w}{ }\PY{+w}{ }\PY{n}{\PYZus{}v\PYZus{}26}\PY{+w}{ }\PY{o}{=}\PY{+w}{ }\PY{l+m+mi}{0}\PY{+w}{ }\PY{k}{fby}\PY{+w}{ }\PY{n}{o}
\PY{k}{tel}
\end{Verbatim}


  Le code est normalisé et ordonnancé.
\end{frame}

\begin{frame}
  \frametitle{Exemple 1 : compteur - code VHDL}

  \tiny

  \begin{columns}[t]
    \hspace{1cm}
    \begin{column}{6cm}
      \begin{Verbatim}[commandchars=\\\{\}]
\PY{k+kn}{use}\PY{+w}{ }\PY{n}{work}\PY{o}{.}\PY{n}{compteur}\PY{o}{.}\PY{n}{all}\PY{p}{;}

\PY{k+kn}{library}\PY{+w}{ }\PY{n}{ieee}\PY{p}{;}
\PY{k+kn}{use}\PY{+w}{ }\PY{n}{ieee}\PY{o}{.}\PY{k+kt}{std\PYZus{}logic}\PY{n}{\PYZus{}1164}\PY{o}{.}\PY{n}{all}\PY{p}{;}

\PY{k+kd}{entity}\PY{+w}{ }\PY{n}{compteur}\PY{+w}{ }\PY{n}{is}
\PY{+w}{ }\PY{+w}{ }\PY{k}{port}\PY{+w}{ }\PY{p}{(}\PY{k+kd}{signal}\PY{+w}{ }\PY{n}{clk\PYZus{}1}\PY{+w}{ }\PY{p}{:}\PY{+w}{ }\PY{n}{in}\PY{+w}{ }\PY{k+kt}{std\PYZus{}logic}\PY{p}{;}\PY{+w}{ }
\PY{+w}{ }\PY{+w}{ }\PY{+w}{ }\PY{+w}{ }\PY{+w}{ }\PY{+w}{ }\PY{+w}{ }\PY{+w}{ }\PY{k+kd}{signal}\PY{+w}{ }\PY{n}{hw\PYZus{}rst\PYZus{}3}\PY{+w}{ }\PY{p}{:}\PY{+w}{ }\PY{n}{in}\PY{+w}{ }\PY{k+kt}{std\PYZus{}logic}\PY{p}{;}
\PY{+w}{ }\PY{+w}{ }\PY{+w}{ }\PY{+w}{ }\PY{+w}{ }\PY{+w}{ }\PY{+w}{ }\PY{+w}{ }\PY{k+kd}{signal}\PY{+w}{ }\PY{n}{rst\PYZus{}2}\PY{+w}{ }\PY{p}{:}\PY{+w}{ }\PY{n}{in}\PY{+w}{ }\PY{k+kt}{std\PYZus{}logic}\PY{p}{;}\PY{+w}{ }
\PY{+w}{ }\PY{+w}{ }\PY{+w}{ }\PY{+w}{ }\PY{+w}{ }\PY{+w}{ }\PY{+w}{ }\PY{+w}{ }\PY{k+kd}{signal}\PY{+w}{ }\PY{n}{tick}\PY{+w}{ }\PY{p}{:}\PY{+w}{ }\PY{n}{in}\PY{+w}{ }\PY{k+kt}{std\PYZus{}logic}\PY{p}{;}
\PY{+w}{ }\PY{+w}{ }\PY{+w}{ }\PY{+w}{ }\PY{+w}{ }\PY{+w}{ }\PY{+w}{ }\PY{+w}{ }\PY{k+kd}{signal}\PY{+w}{ }\PY{n}{top}\PY{+w}{ }\PY{p}{:}\PY{+w}{ }\PY{n}{in}\PY{+w}{ }\PY{k+kt}{std\PYZus{}logic}\PY{p}{;}
\PY{+w}{ }\PY{+w}{ }\PY{+w}{ }\PY{+w}{ }\PY{+w}{ }\PY{+w}{ }\PY{+w}{ }\PY{+w}{ }\PY{k+kd}{signal}\PY{+w}{ }\PY{n}{o\PYZus{}o}\PY{+w}{ }\PY{p}{:}\PY{+w}{ }\PY{n}{out}\PY{+w}{ }\PY{k+kt}{integer}\PY{p}{)}\PY{p}{;}
\PY{k}{end}\PY{+w}{ }\PY{k+kd}{entity}\PY{+w}{ }\PY{n}{compteur}\PY{p}{;}

\PY{k+kd}{architecture}\PY{+w}{ }\PY{n}{rtl}\PY{+w}{ }\PY{k}{of}\PY{+w}{ }\PY{n}{compteur}\PY{+w}{ }\PY{n}{is}
\PY{+w}{ }\PY{+w}{ }\PY{k+kd}{signal}\PY{+w}{ }\PY{n}{h\PYZus{}v\PYZus{}26}\PY{+w}{ }\PY{p}{:}\PY{+w}{ }\PY{k+kt}{integer}\PY{p}{;}
\PY{k}{begin}
\PY{+w}{ }\PY{+w}{ }\PY{n}{update}\PY{+w}{ }\PY{p}{:}\PY{+w}{ }\PY{k+kd}{process}\PY{+w}{ }\PY{p}{(}\PY{n}{clk\PYZus{}1}\PY{p}{,}\PY{+w}{ }\PY{n}{hw\PYZus{}rst\PYZus{}3}\PY{p}{,}\PY{+w}{ }\PY{n}{rst\PYZus{}2}\PY{p}{,}
\PY{+w}{ }\PY{+w}{ }\PY{+w}{ }\PY{+w}{ }\PY{+w}{ }\PY{+w}{ }\PY{+w}{ }\PY{+w}{ }\PY{+w}{ }\PY{+w}{ }\PY{+w}{ }\PY{+w}{ }\PY{+w}{ }\PY{+w}{ }\PY{+w}{ }\PY{+w}{ }\PY{+w}{ }\PY{+w}{ }\PY{+w}{ }\PY{+w}{ }\PY{n}{tick}\PY{p}{,}\PY{+w}{ }\PY{n}{top}\PY{p}{)}
\PY{+w}{ }\PY{+w}{ }\PY{+w}{ }\PY{+w}{ }\PY{k+kd}{variable}\PY{+w}{ }\PY{n}{h\PYZus{}v\PYZus{}27}\PY{+w}{ }\PY{p}{:}\PY{+w}{ }\PY{k+kt}{integer}\PY{p}{;}
\PY{+w}{ }\PY{+w}{ }\PY{+w}{ }\PY{+w}{ }\PY{k+kd}{variable}\PY{+w}{ }\PY{n}{h\PYZus{}v\PYZus{}28}\PY{+w}{ }\PY{p}{:}\PY{+w}{ }\PY{k+kt}{integer}\PY{p}{;}
\PY{+w}{ }\PY{+w}{ }\PY{+w}{ }\PY{+w}{ }\PY{k+kd}{variable}\PY{+w}{ }\PY{n}{o}\PY{+w}{ }\PY{p}{:}\PY{+w}{ }\PY{k+kt}{integer}\PY{p}{;}
\PY{+w}{ }\PY{+w}{ }\PY{k}{begin}
\PY{+w}{ }\PY{+w}{ }\PY{+w}{ }\PY{+w}{ }\PY{n}{case}\PY{+w}{ }\PY{n}{top}\PY{+w}{ }\PY{n}{is}
\PY{+w}{ }\PY{+w}{ }\PY{+w}{ }\PY{+w}{ }\PY{+w}{ }\PY{+w}{ }\PY{k}{when}\PY{+w}{ }\PY{p}{'}\PY{l+m+mi}{1}\PY{p}{'}\PY{+w}{ }\PY{p}{=>}\PY{+w}{ }\PY{n}{h\PYZus{}v\PYZus{}27}\PY{+w}{ }\PY{p}{:}\PY{o}{=}\PY{+w}{ }\PY{l+m+mi}{0}\PY{p}{;}
\PY{+w}{ }\PY{+w}{ }\PY{+w}{ }\PY{+w}{ }\PY{+w}{ }\PY{+w}{ }\PY{k}{when}\PY{+w}{ }\PY{p}{'}\PY{l+m+mi}{0}\PY{p}{'}\PY{+w}{ }\PY{p}{=>}\PY{+w}{ }\PY{n}{case}\PY{+w}{ }\PY{n}{rst\PYZus{}2}\PY{+w}{ }\PY{n}{is}
\PY{+w}{ }\PY{+w}{ }\PY{+w}{ }\PY{+w}{ }\PY{+w}{ }\PY{+w}{ }\PY{+w}{ }\PY{+w}{ }\PY{+w}{ }\PY{+w}{ }\PY{+w}{ }\PY{+w}{ }\PY{+w}{ }\PY{+w}{ }\PY{+w}{ }\PY{+w}{ }\PY{+w}{ }\PY{+w}{ }\PY{+w}{ }\PY{+w}{ }\PY{k}{when}\PY{+w}{ }\PY{p}{'}\PY{l+m+mi}{1}\PY{p}{'}\PY{+w}{ }\PY{p}{=>}\PY{+w}{ }\PY{n}{h\PYZus{}v\PYZus{}27}\PY{+w}{ }\PY{p}{:}\PY{o}{=}\PY{+w}{ }\PY{l+m+mi}{0}\PY{p}{;}
\PY{+w}{ }\PY{+w}{ }\PY{+w}{ }\PY{+w}{ }\PY{+w}{ }\PY{+w}{ }\PY{+w}{ }\PY{+w}{ }\PY{+w}{ }\PY{+w}{ }\PY{+w}{ }\PY{+w}{ }\PY{+w}{ }\PY{+w}{ }\PY{+w}{ }\PY{+w}{ }\PY{+w}{ }\PY{+w}{ }\PY{+w}{ }\PY{+w}{ }\PY{k}{when}\PY{+w}{ }\PY{p}{'}\PY{l+m+mi}{0}\PY{p}{'}\PY{+w}{ }\PY{p}{=>}\PY{+w}{ }\PY{n}{h\PYZus{}v\PYZus{}27}\PY{+w}{ }\PY{p}{:}\PY{o}{=}\PY{+w}{ }\PY{n}{h\PYZus{}v\PYZus{}26}\PY{p}{;}
\PY{+w}{ }\PY{+w}{ }\PY{+w}{ }\PY{+w}{ }\PY{+w}{ }\PY{+w}{ }\PY{+w}{ }\PY{+w}{ }\PY{+w}{ }\PY{+w}{ }\PY{+w}{ }\PY{+w}{ }\PY{+w}{ }\PY{+w}{ }\PY{+w}{ }\PY{+w}{ }\PY{+w}{ }\PY{+w}{ }\PY{k}{end}\PY{+w}{ }\PY{n}{case}\PY{p}{;}
\PY{+w}{ }\PY{+w}{ }\PY{+w}{ }\PY{+w}{ }\PY{k}{end}\PY{+w}{ }\PY{n}{case}\PY{p}{;}
\end{Verbatim}

    \end{column}
    \begin{column}{6cm}
      \begin{Verbatim}[commandchars=\\\{\}]
\PY{+w}{ }\PY{+w}{ }\PY{+w}{ }\PY{+w}{ }\PY{n}{case}\PY{+w}{ }\PY{n}{tick}\PY{+w}{ }\PY{n}{is}
\PY{+w}{ }\PY{+w}{ }\PY{+w}{ }\PY{+w}{ }\PY{+w}{ }\PY{+w}{ }\PY{k}{when}\PY{+w}{ }\PY{p}{'}\PY{l+m+mi}{1}\PY{p}{'}\PY{+w}{ }\PY{p}{=>}\PY{+w}{ }\PY{n}{h\PYZus{}v\PYZus{}28}\PY{+w}{ }\PY{p}{:}\PY{o}{=}\PY{+w}{ }\PY{l+m+mi}{1}\PY{p}{;}
\PY{+w}{ }\PY{+w}{ }\PY{+w}{ }\PY{+w}{ }\PY{+w}{ }\PY{+w}{ }\PY{k}{when}\PY{+w}{ }\PY{p}{'}\PY{l+m+mi}{0}\PY{p}{'}\PY{+w}{ }\PY{p}{=>}\PY{+w}{ }\PY{n}{h\PYZus{}v\PYZus{}28}\PY{+w}{ }\PY{p}{:}\PY{o}{=}\PY{+w}{ }\PY{l+m+mi}{0}\PY{p}{;}
\PY{+w}{ }\PY{+w}{ }\PY{+w}{ }\PY{+w}{ }\PY{k}{end}\PY{+w}{ }\PY{n}{case}\PY{p}{;}
\PY{+w}{ }\PY{+w}{ }\PY{+w}{ }\PY{+w}{ }\PY{n}{o}\PY{+w}{ }\PY{p}{:}\PY{o}{=}\PY{+w}{ }\PY{p}{(}\PY{n}{h\PYZus{}v\PYZus{}27}\PY{+w}{ }\PY{o}{+}\PY{+w}{ }\PY{n}{h\PYZus{}v\PYZus{}28}\PY{p}{)}\PY{p}{;}
\PY{+w}{ }\PY{+w}{ }\PY{+w}{ }\PY{+w}{ }\PY{k}{if}\PY{+w}{ }\PY{p}{(}\PY{n}{hw\PYZus{}rst\PYZus{}3}\PY{+w}{ }\PY{o}{=}\PY{+w}{ }\PY{p}{'}\PY{l+m+mi}{1}\PY{p}{'}\PY{p}{)}\PY{+w}{ }\PY{k}{then}
\PY{+w}{ }\PY{+w}{ }\PY{+w}{ }\PY{+w}{ }\PY{+w}{ }\PY{+w}{ }\PY{n}{h\PYZus{}v\PYZus{}26}\PY{+w}{ }\PY{o}{<=}\PY{+w}{ }\PY{l+m+mi}{0}\PY{p}{;}
\PY{+w}{ }\PY{+w}{ }\PY{+w}{ }\PY{+w}{ }\PY{k}{elsif}\PY{+w}{ }\PY{n}{rising\PYZus{}edge}\PY{p}{(}\PY{n}{clk\PYZus{}1}\PY{p}{)}\PY{+w}{ }\PY{k}{then}
\PY{+w}{ }\PY{+w}{ }\PY{+w}{ }\PY{+w}{ }\PY{+w}{ }\PY{+w}{ }\PY{n}{h\PYZus{}v\PYZus{}26}\PY{+w}{ }\PY{o}{<=}\PY{+w}{ }\PY{n}{o}\PY{p}{;}
\PY{+w}{ }\PY{+w}{ }\PY{+w}{ }\PY{+w}{ }\PY{k}{end}\PY{+w}{ }\PY{k}{if}\PY{p}{;}
\PY{+w}{ }\PY{+w}{ }\PY{+w}{ }\PY{+w}{ }\PY{n}{o\PYZus{}o}\PY{+w}{ }\PY{o}{<=}\PY{+w}{ }\PY{n}{o}\PY{p}{;}
\PY{+w}{ }\PY{+w}{ }\PY{k}{end}\PY{+w}{ }\PY{k+kd}{process}\PY{+w}{ }\PY{n}{update}\PY{p}{;}
\PY{k}{end}\PY{+w}{ }\PY{k+kd}{architecture}\PY{+w}{ }\PY{n}{rtl}\PY{p}{;}
\end{Verbatim}

    \end{column}
  \end{columns}
\end{frame}

\begin{frame}
  \frametitle{Exemple 1 : compteur - caractéristiques du code VHDL}

  \begin{block}{Simulation comportementale}
    \vspace{0.3cm}
    \includegraphics[width=10cm]{capture-chrono-compteur}
  \end{block}

  \begin{block}{Synthèse}
    L'outil industriel \textit{Xilinx XST} synthétise une netlist avec des
    caractéristiques satisfaisantes :

    \begin{itemize}
    \item Un additionneur 32 bits pour \textbf{o}.
    \item Un registre 32 bits pour \textbf{h\_v\_26}.
    \end{itemize}
  \end{block}
\end{frame}

\end{document}
